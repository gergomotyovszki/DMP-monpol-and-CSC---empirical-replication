

    \filetitle{freq}{Date frequency of tseries object}{tseries/freq}

	\paragraph{Syntax}

\begin{verbatim}
F = freq(X)
\end{verbatim}

\paragraph{Input arguments}

\begin{itemize}
\itemsep1pt\parskip0pt\parsep0pt
\item
  \texttt{X} {[} tseries {]} - Input tseries object.
\end{itemize}

\paragraph{Output arguments}

\begin{itemize}
\itemsep1pt\parskip0pt\parsep0pt
\item
  \texttt{F} {[} \texttt{0} \textbar{} \texttt{1} \textbar{} \texttt{2}
  \textbar{} \texttt{4} \textbar{} \texttt{6} \textbar{} \texttt{12}
  \textbar{} \texttt{52} \textbar{} \texttt{365} {]} - Date frequency of
  observations in the input tseries object; \texttt{F} is the number of
  periods within a year.
\end{itemize}

\paragraph{Description}

The \texttt{freq( )} function is equivalent to calling the
\texttt{get( )} function:

\begin{verbatim}
get(x,'freq')
\end{verbatim}

\paragraph{Example}


