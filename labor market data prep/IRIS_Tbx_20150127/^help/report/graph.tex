

    \filetitle{graph}{Add graph to figure}{report/graph}

	\paragraph{Syntax}

\begin{verbatim}
P.graph(Cap,...)
\end{verbatim}

\paragraph{Input arguments}

\begin{itemize}
\item
  \texttt{P} {[} struct {]} - Report object created by the
  \href{report/new}{\texttt{report.new}} function.
\item
  \texttt{Cap} {[} char \textbar{} cellstr \textbar{} \texttt{@auto} {]}
  - Title, or cell array with title and subtitle, displayed at the top
  of the graph; \texttt{@auto} means that the first comment from the
  first child \texttt{series} object will be used for the title.
\end{itemize}

\paragraph{Options}

\begin{itemize}
\item
  \texttt{'axesOptions='} {[} cell \textbar{} \emph{empty} {]} -
  (Inheritable) Options executed by calling \texttt{set} on the axes
  handle before running \texttt{'postProcess='}.
\item
  \texttt{'dateTick='} {[} numeric \textbar{} \emph{\texttt{Inf}} {]} -
  (Inheritable) Date tick spacing.
\item
  \texttt{'grid='} {[} \texttt{@auto} \textbar{} \texttt{true}
  \textbar{} \texttt{false} {]} - (Inheritable) Display grid lines; if
  \texttt{@auto}, \texttt{'grid='} is \texttt{true} unless a
  right-hand-side axis is plotted.
\item
  \texttt{'legend='} {[} \emph{\texttt{false}} \textbar{} \texttt{true}
  {]} - (Inheritable) Add legend to the graph.
\item
  \texttt{'legendLocation='} {[} char \textbar{} \emph{\texttt{'best'}}
  \textbar{} \texttt{'bottom'}{]} - (Inheritable) Location of the legend
  box; see help on \texttt{legend} for values available.
\item
  \texttt{'postProcess='} {[} char \textbar{} \emph{empty} {]} -
  (Inheritable) String with Matlab commands executed after the graph has
  been drawn and styled; see Description.
\item
  \texttt{'preProcess='} {[} char \textbar{} \emph{empty} {]} -
  (Inheritable) String with Matlab commands executed before the graph
  has been drawn and styled; see Description.
\item
  \texttt{'range='} {[} numeric \textbar{} \emph{\texttt{Inf}} {]} -
  (Inheritable) Graph range.
\item
  \texttt{'rhsAxesOptions='} {[} cell \textbar{} \emph{empty} {]} -
  (Inheritable) Options executed by calling \texttt{set} on the RHS axes
  handle before running \texttt{'postProcess='}.
\item
  \texttt{'style='} {[} struct \textbar{} \emph{empty} {]} -
  (Inheritable) Apply this style structure to the graph and its
  children; see help on \href{qreport/qstyle}{\texttt{qstyle}}.
\item
  \texttt{'tight='} {[} \texttt{@auto} \textbar{} \texttt{true}
  \textbar{} \texttt{false} {]} - (Inheritable) Set the y-axis limits to
  the minimum and maximum of displayed data; if \texttt{@auto},
  \texttt{'tight='} is \texttt{true} unless a right-hand-side axis is
  plotted.
\item
  \texttt{'xLabel='} {[} char \textbar{} \emph{empty} {]} - Label the
  x-axis.
\item
  \texttt{'yLabel='} {[} char \textbar{} \emph{empty} {]} - Label the
  y-axis.
\item
  \texttt{'zeroLine='} {[} \texttt{true} \textbar{}
  \emph{\texttt{false}} \textbar{} cell {]} - (Inheritable) Add a
  horizontal zero line if zero is included on the y-axis; specify
  zeroline options in a cell array.
\end{itemize}

\paragraph{Date format options}

See \href{dates/dat2str}{\texttt{dat2str}} for details on date format
options.

\begin{itemize}
\item
  \texttt{'dateFormat='} {[} char \textbar{} cellstr \textbar{}
  \emph{\texttt{'YYYYFP'}} {]} - Date format string, or array of format
  strings (possibly different for each date).
\item
  \texttt{'freqLetters='} {[} char \textbar{} \emph{\texttt{'YHQBMW'}}
  {]} - Six letters used to represent the six possible frequencies of
  IRIS dates, in this order: yearly, half-yearly, quarterly, bi-monthly,
  monthly, and weekly (such as the \texttt{'Q'} in \texttt{'2010Q1'}).
\item
  \texttt{'months='} {[} cellstr \textbar{}
  \emph{\texttt{\{'January',...,'December'\}}} {]} - Twelve strings
  representing the names of the twelve months.
\item
  \texttt{'standinMonth='} {[} numeric \textbar{} \texttt{'last'}
  \textbar{} \emph{\texttt{1}} {]} - Month that will represent a
  lower-than-monthly-frequency date if the month is part of the date
  format string.
\end{itemize}

\paragraph{Generic options}

See help on \href{report/Contents}{generic options} in report objects.

\paragraph{Description}

The options \texttt{'preProcess='} and \texttt{'postProcess='} give you
additional flexibility in customising the graphics style of the axes
object. The values assigned to these options are expected to be strings
with an executable Matlab command, or commands separated with
semi-colons (as if typed on one line in the command window). The command
can refer to the following variables:

\begin{itemize}
\itemsep1pt\parskip0pt\parsep0pt
\item
  \texttt{H} - a handle to the currently processed axes object.
\item
  \texttt{L} - a handle to the corresponding legend object; if no legend
  object exists for the axes \texttt{H}, \texttt{L} will be
  \texttt{NaN}.
\end{itemize}

\paragraph{Example}

Create a one-page report with a chart in on the LHS and the legend moved
to the RHS. Use the function \texttt{grfun.movetosubplot} in the option
\texttt{'postProcess='}, referring to \texttt{L} (handle to the legend
object associated with the respective axes object) to move the legend
around.

\begin{verbatim}
% Create random data series.
A = tseries(1:10,@rand);
B = tseries(1:10,@rand);

% Open a new report.
x = report.new();

% Open a new figure in the report with a 1-by-2 layout.
x.figure('My Figure','subplot=',[1,2]);

    % The graph will be placed in the LHS space.
    % Use `grfun.movetosubplot` to move the legend to the RHS space.
    x.graph('My Graph','legend=',true, ...
        'postProcess=','grfun.movetosubplot(L,1,2,2)');

        x.series('Series A',A);
        x.series('Series B',B);

x.publish('test.pdf');
open test.pdf;
\end{verbatim}


