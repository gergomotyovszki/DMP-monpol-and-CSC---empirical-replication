

    \filetitle{dbnames}{List of database entries filtered by name and/or class}{dbase/dbnames}

	\paragraph{Syntax}

\begin{verbatim}
List = dbnames(D,...)
\end{verbatim}

\paragraph{Input arguments}

\begin{itemize}
\itemsep1pt\parskip0pt\parsep0pt
\item
  \texttt{D} {[} struct {]} - Input database.
\end{itemize}

\paragraph{Output arguments}

\begin{itemize}
\itemsep1pt\parskip0pt\parsep0pt
\item
  \texttt{List} {[} cellstr {]} - List of input database entries that
  pass the name or class test.
\end{itemize}

\paragraph{Options}

\begin{itemize}
\item
  \texttt{'nameFilter='} {[} char \textbar{} \emph{\texttt{Inf}} {]} -
  Regular expression against which the database entry names will be
  matched; \texttt{Inf} means all names will be matched.
\item
  \texttt{'classFilter='} {[} char \textbar{} \emph{\texttt{Inf}} {]} -
  Regular expression against which the database entry class names will
  be matched; \texttt{Inf} means all classes will be matched.
\end{itemize}

\paragraph{Description}

\paragraph{Example}

Notice the differences in the following calls to \texttt{dbnames}:

\begin{verbatim}
dbnames(d,'nameFilter=','L_')
\end{verbatim}

matches all names that contain \texttt{'L\_'} (at the beginning, in the
middle, or at the end of the string), such as \texttt{'L\_A'},
\texttt{'DL\_A'}, \texttt{'XL\_'}, or just \texttt{'L\_'}.

\begin{verbatim}
dbnames(d,'nameFilter=','^L_')
\end{verbatim}

matches all names that start with \texttt{'L\_'}, suc as \texttt{'L\_A'}
or \texttt{'L\_'}.

\begin{verbatim}
dbnames(d,'nameFilter=','^L_.')
\end{verbatim}

matches all names that start with \texttt{'L\_'} and have at least one
more character after that, such as \texttt{'L\_A'} (but not
\texttt{'L\_'}).


