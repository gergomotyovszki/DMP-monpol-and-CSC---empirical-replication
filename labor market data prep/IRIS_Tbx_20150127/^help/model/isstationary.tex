

    \filetitle{isstationary}{True if model or specified combination of variables is stationary}{model/isstationary}

	\paragraph{Syntax}

\begin{verbatim}
Flag = isstationary(M)
Flag = isstationary(M,TName)
Flag = isstationary(M,TLinComb)
\end{verbatim}

\paragraph{Input arguments}

\begin{itemize}
\item
  \texttt{M} {[} model {]} - Model object.
\item
  \texttt{TName} {[} char {]} - Name of a transition variable.
\item
  \texttt{Expn} {[} char {]} - Text string defining a linear combination
  of transition variables; log variables need to be enclosed in
  \texttt{log(...)}.
\end{itemize}

\paragraph{Output arguments}

\begin{itemize}
\itemsep1pt\parskip0pt\parsep0pt
\item
  \texttt{Flag} {[} \texttt{true} \textbar{} \texttt{false} {]} - True
  if the model (if called without a second input argument) or the
  specified transition variable or combination of transition variables
  (if called with a second input argument) is stationary.
\end{itemize}

\paragraph{Description}

\paragraph{Example}

In the following examples, \texttt{m} is a solved model object with two
of its transition variables named \texttt{X} and \texttt{Y}; the latter
is a log variable:

\begin{verbatim}
isstationary(m)
isstationary(m,'X')
isstationary(m,'log(Y)')
isstationary(m,'X - 0.5*log(Y)')
\end{verbatim}


