
    \foldertitle{rpteq}{Reporting Equations (rpteq Objects)}{rpteq/Contents}

	Reporting equations (rpteq) objects are systems of equations evaluated
successively (i.e.~not simultaneously) equation by equation, period by
period.

There are three basic ways to create reporting equations objects:

\begin{itemize}
\item
  in the
  \href{modellang/reportingequations}{\texttt{!reporting\_equations}}
  section of a model file;
\item
  in a separate reporting equations file;
\item
  on the fly within an m-file or in the command window.
\end{itemize}

Rpteq methods:

\paragraph{Constructor}

\begin{itemize}
\itemsep1pt\parskip0pt\parsep0pt
\item
  \href{rpteq/rpteq}{\texttt{rpteq}} - New reporting equations (rpteq)
  object.
\end{itemize}

\paragraph{Evaluating reporting
equations}

\begin{itemize}
\itemsep1pt\parskip0pt\parsep0pt
\item
  \href{rpteq/run}{\texttt{run}} - Evaluate reporting equations (rpteq)
  object.
\end{itemize}

\paragraph{Evaluating reporting equations from within model
object}

\begin{itemize}
\itemsep1pt\parskip0pt\parsep0pt
\item
  \href{model/reporting}{\texttt{reporting}} - Evaluate reporting
  equations from within model object.
\end{itemize}

\paragraph{Getting on-line help on rpteq
functions}

\begin{verbatim}
help rpteq
help rpteq/function_name
\end{verbatim}



