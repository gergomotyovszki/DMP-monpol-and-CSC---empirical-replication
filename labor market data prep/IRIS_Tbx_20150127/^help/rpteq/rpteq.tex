

    \filetitle{rpteq}{New reporting equations (rpteq) object}{rpteq/rpteq}

	\paragraph{Syntax}

\begin{verbatim}
Q = rpteq(FName)
Q = rpteq(Eqtn)
\end{verbatim}

\paragraph{Input arguments}

\begin{itemize}
\item
  \texttt{FName} {[} char \textbar{} cellstr {]} - File name or cellstr
  array of file names, each a plain text file with reporting equations;
  multiple input files will be combined together.
\item
  \texttt{Eqtn} {[} char \textbar{} cellstr {]} - Equation or cellstr
  array of equations.
\end{itemize}

\paragraph{Output arguments}

\begin{itemize}
\itemsep1pt\parskip0pt\parsep0pt
\item
  \texttt{Q} {[} rpteq {]} - New reporting equations object.
\end{itemize}

\paragraph{Description}

Reporting equations must be written in the following form:

\begin{verbatim}
`LhsName = RhsExpr;`
`"Label" LhsName = RhsExpr;`
\end{verbatim}

where

\begin{itemize}
\item
  \texttt{LhsName} is the name of a left-hand-side variable (with no lag
  or lead);
\item
  \texttt{RhsExpr} is an expression on the right-hand side that will be
  evaluated period by period, and assigned to the left-hand-side
  variable, \texttt{LhsName}.
\item
  \texttt{"Label"} is an optional label that will be used to create a
  comment in the output time series for the respective left-hand-side
  variable.
\item
  the equation must end with a semicolon.
\end{itemize}

\paragraph{Example}

\begin{verbatim}
q = rpteq({ ...
    'a = c * a{-1}^0.8 * b{-1}^0.2;', ...
    'b = sqrt(b{-1});', ...
    })

q =
    rpteq object
    number of equations: [2]
    comment: ''
    user data: empty
    export files: [0]
\end{verbatim}


